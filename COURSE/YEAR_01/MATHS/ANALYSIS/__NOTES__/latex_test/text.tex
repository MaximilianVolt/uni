\documentclass[12pt,a4paper]{article}

% Pacchetti base
\usepackage[italian]{babel}
\usepackage[utf8]{inputenc}
\usepackage[T1]{fontenc}
\usepackage{lmodern}
\usepackage[a4paper,margin=2.5cm]{geometry}
\usepackage{amsmath,amssymb,amsthm,mathtools}
\usepackage{enumitem}
\usepackage{hyperref}
\usepackage{bookmark}
\usepackage{physics}
\usepackage{siunitx}
\usepackage{microtype}
\usepackage{xcolor}

% Stili
\hypersetup{
colorlinks=true,
linkcolor=blue!60!black,
citecolor=blue!60!black,
urlcolor=blue!60!black
}

\setlist[itemize]{itemsep=2pt,topsep=4pt}
\setlist[enumerate]{itemsep=2pt,topsep=4pt}

% Ambienti personalizzati
\newtheoremstyle{mystyle}
{6pt}{6pt}{}{}{\bfseries}{.}{0.5em}{}
\theoremstyle{mystyle}
\newtheorem{defn}{Definizione}[section]
\newtheorem{prop}[defn]{Proposizione}
\newtheorem{teo}[defn]{Teorema}
\newtheorem{oss}[defn]{Osservazione}

\newenvironment{esempio}[Esempio]{\par\noindent\textbf{#1.}\ }{\par}
\newenvironment{esercizi}[Esercizi]{\par\noindent\textbf{#1.}\ \begin{enumerate}[label=\arabic*]}{\end{enumerate}\par}
\newenvironment{checklist}[Checklist]{\par\noindent\textbf{#1.}\ \begin{itemize}}{\end{itemize}\par}
\newenvironment{errori}[Errori tipici]{\par\noindent\textbf{#1.}\ \begin{itemize}}{\end{itemize}\par}

% Comandi utili
\newcommand{\R}{\mathbb{R}}
\newcommand{\Grad}{\nabla}
\newcommand{\Hess}{\mathrm{H}}
\newcommand{\dx}{,\mathrm{d}x}
\newcommand{\dy}{,\mathrm{d}y}
\newcommand{\dr}{,\mathrm{d}r}
\newcommand{\dt}{,\mathrm{d}\theta}
\newcommand{\dA}{,\mathrm{d}A}
\newcommand{\eps}{\varepsilon}

% Frontespizio
\title{\textbf{Analisi Matematica II — Compendio Operativo Esteso}\
\large Corso di Laurea Triennale in Informatica — UNIPG}
\author{Appunti operativi per la preparazione all'esame orale}
\date{}

\begin{document}
\maketitle
\tableofcontents
\newpage

\section*{Guida d'uso}
Questo compendio è organizzato per massimizzare l'efficacia nello studio:
\begin{itemize}
\item Teoria minima essenziale per fare gli esercizi.
\item Procedure operative passo--passo.
\item Esempi svolti ``stile esame''.
\item Errori tipici da evitare.
\item Checklist finali di autoverifica.
\item Esercizi proposti a difficoltà crescente.
\end{itemize}
Convenzioni: $f_x,f_y$ derivate parziali; $\Grad f$ gradiente; $\Hess f$ Hessiano; $D=\det(\Hess f)$; autovalori $\lambda_1,\lambda_2$.

\section{Massimi e minimi per funzioni di due variabili}
\subsection{Mappa concettuale e teoria minima}
\begin{defn}[Punto critico]
Sia $f:\R^2\to\R$ sufficientemente regolare. Un punto $p=(x_0,y_0)$ è \emph{critico} se $\Grad f(p)=(0,0)$.
\end{defn}

\begin{defn}[Hessiano e test classico]
L'Hessiano è $\Hess f=\begin{pmatrix} f_{xx} & f_{xy}\ f_{yx} & f_{yy}\end{pmatrix}$. Nel punto critico $p$, sia $D=f_{xx}(p)f_{yy}(p)-f_{xy}(p)^2$. Allora:
\begin{itemize}
\item $D>0$ e $f_{xx}(p)>0$: minimo locale.
\item $D>0$ e $f_{xx}(p)<0$: massimo locale.
\item $D<0$: punto di sella.
\item $D=0$: test inconcludente (serve analisi ulteriore).
\end{itemize}
\end{defn}

\begin{oss}[Metodo degli autovalori]
Equivalentemente, valuta i segni degli autovalori di $\Hess f(p)$: entrambi $>0$ (minimo), entrambi $<0$ (massimo), discordi (sella).
\end{oss}

\subsection{Procedura in 7 passi}
\begin{enumerate}
\item Calcola $f_x,f_y$.
\item Risolvi $\Grad f=(0,0)$ per i punti critici.
\item Costruisci $\Hess f$.
\item Valuta $\Hess f$ e $D$ in ciascun punto critico.
\item Applica il test (classico o con autovalori) e classifica.
\item Considera vincoli/bordo del dominio.
\item Concludi con motivazione completa.
\end{enumerate}

\subsection{Esempi svolti}
\begin{esempio}
$f(x,y)=x^2+xy+y^2$. $\Grad f=(2x+y,x+2y)=0 \Rightarrow (0,0)$. $\Hess=\begin{pmatrix}2&1\1&2\end{pmatrix}$, $D=3>0$, $f_{xx}=2>0\Rightarrow$ minimo in $(0,0)$.
\end{esempio}

\begin{esempio}
$f(x,y)=x^2-y^2$. $\Grad f=(2x,-2y)=0 \Rightarrow (0,0)$. $\Hess=\mathrm{diag}(2,-2)$, $D<0\Rightarrow$ sella.
\end{esempio}

\begin{esempio}[Caso $D=0$]
$f(x,y)=(x^2-y^2)^2$. In $(0,0)$: $\Grad f=0$, $\Hess(0,0)=0\Rightarrow D=0$. Analisi d'ordine superiore: $f\ge0$ e nullo su $x=\pm y$ (minimo non isolato).
\end{esempio}

\begin{esempio}[Parametro]
$f(x,y)=x^4+a x^2 y^2+y^4$. Classifica in $(0,0)$ al variare di $a$ discutendo le curvature principali (autovalori).
\end{esempio}

\subsection{Errori e checklist}
\begin{errori}
\item Applicare il test fuori da un punto critico.
\item Ignorare bordo/vincoli del dominio.
\item Conclusioni senza $D$ o senza segno di $f_{xx}$.
\end{errori}
\begin{checklist}
\item $\Grad f$ e punti critici corretti?
\item $\Hess$ e $D$ calcolati nel punto?
\item Test applicato con ipotesi verificate?
\item Bordo/vincoli discussi se presenti?
\item Conclusione chiara (tipo di punto)?
\end{checklist}

\subsection{Esercizi proposti}
\begin{esercizi}
\item $f=x^2+2y^2-2xy$.
\item $f=x^3-3xy^2$.
\item $f=\cos x+\cos y$ su $[-\pi,\pi]^2$.
\item $f=e^{x+y}-x^2-y^2$.
\item $f=x^4+y^4-x^2y^2$.
\item $f=\ln(1+x^2+y^2)$.
\item $f=x^2y^2-x^3-y^3$.
\item $f=(x^2+y^2)^2-a(x^2+y^2)$, discuti in $a$.
\item $f=x^2+y^2$ con $x^2+4y^2\le1$ (bordo).
\item $f=x^2-xy+y^2$ su dominio triangolare.
\end{esercizi}

\section{Integrali doppi: cartesiane, polari, cambio d'ordine, Jacobiano}
\subsection{Teoria minima e procedure}
\begin{defn}[Domini normali]
$x$--normale: $R={(x,y): a\le x\le b,\ \phi_1(x)\le y\le \phi_2(x)}$. Analogo per $y$--normale.
\end{defn}
Passaggio a polari: $x=r\cos\theta$, $y=r\sin\theta$, $\dx\dy=r,\dr\dt$.

\paragraph{Procedura generale}
\begin{enumerate}
\item Disegna la regione $R$.
\item Riconosci $x$-- o $y$--normalità; scrivi i limiti coerenti.
\item Valuta cambio d'ordine se semplifica.
\item Valuta coordinate polari (dischi/settori/annulli).
\item Imposta e calcola.
\end{enumerate}

\subsection{Esempi svolti}
\begin{esempio}[Area in cartesiane]
$R$ tra $y=0$ e $y=1-x^2$, $x\in$. Area $=\int_0^1\int_0^{1-x^2} 1,\dy,\dx$.
\end{esempio}

\begin{esempio}[Disco in polari]
$R:\ x^2+y^2\le a^2$, $\iint_R (x^2+y^2),\dA=\int_0^{2\pi}\int_0^{a} r^2\cdot r,\dr,\dt=\frac{2\pi a^4}{4}$.
\end{esempio}

\begin{esempio}[Settore anulare]
$R={b\le r\le a,\ \alpha\le \theta\le \beta}$, integranda radiale $g(r)$: immediato in polari.
\end{esempio}

\begin{esempio}[Cambio d'ordine]
Regione tra $y=\sqrt{x}$ e $y=1$, $x\in$: ridisegna e riscrivi $x$ in funzione di $y$ per semplificare l'interno.
\end{esempio}

\subsection{Errori e checklist}
\begin{errori}
\item Dimenticare il Jacobiano $r$ in polari.
\item Limiti incoerenti; regione non disegnata.
\item Cambio d'ordine che non copre tutta $R$.
\end{errori}
\begin{checklist}
\item Schizzo corretto di $R$?
\item Limiti coerenti con la normalità scelta?
\item Scelta coordinate motivata?
\item Calcolo pulito e controllato?
\end{checklist}

\subsection{Esercizi proposti}
\begin{esercizi}
\item Area di $x^2+y^2\le4$, $y\ge0$.
\item $\iint_R (x+y),\dA$ tra $y=x$ e $y=\sqrt{x}$, $x\in$.
\item Anello $1\le r\le 2$: integra $r^3$.
\item Cambio d'ordine per regione tra $y=x^2$ e $y=2-x$.
\item Funzione radiale su settore $0\le\theta\le\pi/3$.
\item Momento d'inerzia di disco uniforme.
\item Semicerchio: $x\in$, $y\in[0,\sqrt{2x-x^2}]$.
\item $\iint_{\R^2} e^{-(x^2+y^2)},\dA$ (polari).
\item Regione con $y=\ln(1+x)$: discuti normalità.
\item Regione ``a goccia'': più cambi d'ordine.
\end{esercizi}

\section{Impropri, Probability Integral, Funzione Gamma, Gaussiana}
\subsection{Teoria minima}
Impropri per intervalli infiniti o discontinuità: definizioni a limite; criteri di confronto.

\paragraph{Probability Integral}
$I=\int_{-\infty}^{+\infty} e^{-x^2}\dx$. Allora

I
2
=
∬
R
2
e
−
(
x
2
+
y
2
)
 
\dA
=
∫
0
2
π
∫
0
∞
e
−
r
2
 
r
 
\dr
 
\dt
=
2
π
⋅
1
2
=
π
⇒
I
=
π
.
I 
2
 =∬ 
R 
2
 
 e 
−(x 
2
 +y 
2
 )
 \dA=∫ 
0
2π
 ∫ 
0
∞
 e 
−r 
2
 
 r\dr\dt=2π⋅ 
2
1
 =π⇒I= 
π
 .
\paragraph{Funzione Gamma}
$\Gamma(s)=\int_0^\infty x^{s-1}e^{-x}\dx$, $\Gamma(s+1)=s,\Gamma(s)$, $\Gamma(1/2)=\sqrt{\pi}$, $n!=\Gamma(n+1)$.
Per $a>0$: $\int_0^\infty x^{s-1}e^{-ax}\dx=\Gamma(s)/a^s$.

\paragraph{Gaussiana $\mathcal N(0,1)$}
$\varphi(x)=\frac{1}{\sqrt{2\pi}}e^{-x^2/2}$, $\int \varphi=1$ (normalizzazione), media $0$, varianza $1$.

\subsection{Esempi svolti}
\begin{esempio}
$\int_1^\infty \frac{1}{x^p}\dx$ converge $\Leftrightarrow p>1$.
\end{esempio}
\begin{esempio}
$\int_0^1 \frac{1}{\sqrt{x}}\dx$ converge (esponente $<1$).
\end{esempio}
\begin{esempio}
Probability integral via polari (derivazione completa sopra).
\end{esempio}
\begin{esempio}
$\int_0^\infty x^{s-1}e^{-ax}\dx=\Gamma(s)/a^s$ via $y=ax$.
\end{esempio}
\begin{esempio}
Normalizzazione di $\mathcal N(0,1)$ riducendo a $I=\sqrt{\pi}$.
\end{esempio}

\subsection{Errori e checklist}
\begin{errori}
\item Non separare l'integrale su discontinuità.
\item Gestire limiti senza confronto/giustificazione.
\item Dimenticare fattori di normalizzazione nelle densità.
\end{errori}
\begin{checklist}
\item Tipo di improprio identificato?
\item Metodo scelto (confronto/Gamma/polari) adeguato?
\item Passaggi con limiti espliciti?
\item Valore finale corretto?
\end{checklist}

\subsection{Esercizi proposti}
\begin{esercizi}
\item $\int_0^\infty e^{-ax}\dx$ ($a>0$).
\item $\int_0^\infty x^{k} e^{-x}\dx$ ($k\in\mathbb{N}$).
\item $\int_0^\infty x^{-1/2} e^{-x}\dx$.
\item $\int_{-\infty}^{\infty} e^{-(x^2+bx+c)}\dx$ (completa il quadrato).
\item $\int_0^\infty x^{s-1} e^{-ax}\cos(bx)\dx$ (accenno avanzato).
\item Normalizzazione $\mathcal N(\mu,\sigma^2)$.
\item Convergenza di $\int_2^\infty \frac{\ln x}{x^p}\dx$.
\item $\int_0^1 \frac{\dx}{x^\alpha(1-x)^\beta}$: condizioni su $\alpha,\beta$.
\end{esercizi}

\section{Serie numeriche e criteri}
\subsection{Strategia operativa e teoria minima}
\paragraph{Strategia}
\begin{itemize}
\item Fattoriali/$a^n$: Rapporto o Radice.
\item Log lenti: Cauchy (condensazione).
\item Segno alterno + decrescenza: Leibniz; valuta convergenza assoluta.
\item Confronto/asintotico con serie note.
\end{itemize}
\paragraph{Criteri}
Confronto, confronto asintotico, rapporto, radice, condensazione di Cauchy, Leibniz; convergenza assoluta $\Rightarrow$ convergenza.

\subsection{Esempi svolti}
\begin{esempio}
$\sum \frac{1}{n^p}$: soglia $p=1$.
\end{esempio}
\begin{esempio}
$\sum a^n$ con $|a|<1$: geometrica.
\end{esempio}
\begin{esempio}
$\sum \frac{n!}{n^n}$: Rapporto $\to0$ (assoluta).
\end{esempio}
\begin{esempio}
$\sum \frac{(-1)^n}{n\ln n}$: Leibniz converge; assoluta? $\sum 1/(n\ln n)$ diverge $\Rightarrow$ solo condizionata.
\end{esempio}
\begin{esempio}
$\sum \frac{\ln n}{n^p}$: confronta con $1/n^{p-\eps}$ per $p>1$.
\end{esempio}

\subsection{Errori e checklist}
\begin{errori}
\item Confronto asintotico senza $\sim$ valido.
\item Dimenticare monotonia in Leibniz.
\item Non distinguere assoluta/condizionata quando serve.
\end{errori}
\begin{checklist}
\item Ipotesi del criterio verificate?
\item Criterio motivato?
\item Conclusione esplicita (assoluta/condizionata/diverge)?
\end{checklist}

\subsection{Esercizi proposti}
\begin{esercizi}
\item $\sum \frac{n^2}{3^n}$.
\item $\sum \frac{(-1)^n n}{n^2+1}$.
\item $\sum \frac{1}{n(\ln n)^p}$ (regimi in $p$).
\item $\sum \frac{(n!)^2}{(2n)!}$.
\item $\sum \frac{\ln n}{n}$.
\item $\sum \frac{(-1)^n}{\sqrt{n}}$.
\item $\sum \frac{n^a}{b^n}$ ($b>1$).
\item $\sum \frac{1}{n^p(\ln n)^q}$: regioni.
\item $\sum (-1)^n \left(1+\frac{1}{n}\right)^n$.
\item $\sum \frac{n^n}{(n!)^2}$.
\end{esercizi}

\section{Serie di Taylor e limiti}
\subsection{Sviluppi base (intorno a $0$)}

e
x
=
1
+
x
+
x
2
2
!
+
⋯
 
,
sin
⁡
x
=
x
−
x
3
3
!
+
⋯
 
,
cos
⁡
x
=
1
−
x
2
2
!
+
⋯
 
,
e 
x
 =1+x+ 
2!
x 
2
 
 +⋯,sinx=x− 
3!
x 
3
 
 +⋯,cosx=1− 
2!
x 
2
 
 +⋯,
ln
⁡
(
1
+
x
)
=
x
−
x
2
2
+
x
3
3
−
⋯
  
(
∣
x
∣
<
1
)
.
ln(1+x)=x− 
2
x 
2
 
 + 
3
x 
3
 
 −⋯  (∣x∣<1).
\subsection{Procedura per limiti}
\begin{enumerate}
\item Scegli il punto di sviluppo (spesso $0$).
\item Tronca all'ordine necessario (controlla il denominatore).
\item Sostituisci e semplifica (equivalenti asintotici).
\item Coerenza nelle sostituzioni in composti.
\end{enumerate}

\subsection{Esempi svolti}
\begin{esempio}
$\lim_{x\to0}\frac{1-\cos x}{x^2}=\frac{1}{2}$.
\end{esempio}
\begin{esempio}
$\lim_{x\to0}\frac{\ln(1+x)}{x}=1$.
\end{esempio}
\begin{esempio}
$\lim_{x\to0}\frac{e^x-1-x}{x^2}=\frac{1}{2}$.
\end{esempio}
\begin{esempio}
$\sqrt{1+x}=1+\frac{x}{2}-\frac{x^2}{8}+\cdots$ (usa per limiti con radicali).
\end{esempio}
\begin{esempio}
$\frac{\sin(x^2)}{x^2}\to1$; $\frac{\ln(1+ax)}{x}\to a$.
\end{esempio}

\subsection{Errori e checklist}
\begin{errori}
\item Sviluppo fuori raggio.
\item Ordine insufficiente.
\item Sostituzioni incoerenti nei composti.
\end{errori}
\begin{checklist}
\item Sviluppo corretto?
\item Ordine adeguato al problema?
\item Semplificazioni coerenti?
\item Risultato motivato?
\end{checklist}

\subsection{Esercizi proposti}
\begin{esercizi}
\item $\lim_{x\to0}\frac{\tan x - x}{x^3}$.
\item $\lim_{x\to0}\frac{\ln(1+x)-x+x^2/2}{x^3}$.
\item $\lim_{x\to0}\frac{e^{ax}-1-ax}{x^2}$ ($a\in\R$).
\item $\lim_{x\to0}\frac{1-\cos(bx)}{x^2}$ ($b\in\R$).
\item $\lim_{x\to0}\frac{\sqrt{1+x}-1-x/2}{x^2}$.
\item $\lim_{x\to0}\frac{\sin x - x + x^3/6}{x^5}$.
\item $\lim_{x\to0}\frac{\ln\frac{1+\sin x}{1-\sin x}}{x}$.
\item $\lim_{x\to0}\frac{e^{x}-\cos x - x - x^2/2}{x^3}$.
\item $\lim_{x\to0}\frac{\arctan x - x + x^3/3}{x^5}$.
\item $\lim_{x\to0}\frac{\ln(1+x^2)-x^2}{x^4}$.
\end{esercizi}

\section*{Appendice A — Errori tipici trasversali}
\begin{itemize}
\item Integrali doppi senza disegno della regione.
\item Jacobiano $r$ dimenticato in polari.
\item Criteri di serie applicati senza ipotesi (monotonia, positività, limiti).
\item Assoluta/condizionata non specificata.
\item Bordo/vincoli ignorati nei massimi/minimi.
\end{itemize}

\section*{Appendice B — Mini--formulario}
\begin{itemize}
\item $\Grad f=(f_x,f_y)$, $\Hess f=\begin{pmatrix} f_{xx} & f_{xy}\ f_{yx} & f_{yy}\end{pmatrix}$, $D=f_{xx}f_{yy}-f_{xy}^2$.
\item Polari: $x=r\cos\theta$, $y=r\sin\theta$, $\dA=r,\dr,\dt$.
\item $\Gamma(s)=\int_0^\infty x^{s-1}e^{-x}\dx$, $\Gamma(s+1)=s\Gamma(s)$, $\Gamma(1/2)=\sqrt{\pi}$.
\item $\int_{-\infty}^{\infty} e^{-x^2/2}\dx=\sqrt{2\pi}$.
\item Sviluppi: $e^x$, $\sin x$, $\cos x$, $\ln(1+x)$.
\item Criteri serie: Confronto, Confronto asintotico, Rapporto, Radice, Cauchy, Leibniz.
\end{itemize}

\section*{Appendice C — Checklist finali ``stile orale''}
\begin{itemize}
\item \textbf{Massimi/Minimi:} $\Grad f=0$? $\Hess$, $D$? Bordo discusso? Conclusione motivata?
\item \textbf{Integrali doppi:} Regione disegnata? Limiti coerenti? Coordinate motivate? Calcolo pulito?
\item \textbf{Impropri/Gamma:} Tipo improprio? Limiti espliciti? Normalizzazioni corrette?
\item \textbf{Serie:} Criterio adatto con ipotesi verificate? Esito assoluta/condizionata?
\item \textbf{Taylor:} Sviluppo corretto? Ordine adeguato? Risultato ben giustificato?
\end{itemize}

\end{document}