%%% Math snippets - these sure will spare me tons of time

\usepackage{expl3}
\usepackage{xparse}
\usepackage{pst-plot}
\usepackage{amsmath, amsthm, amsfonts}



%% Basic



% Param #1: the text to highlight as variable

\DeclareRobustCommand{\var}[1] {
  \hlVar{#1}
}



% Param #1: the text to highlight as operator

\DeclareRobustCommand{\op}[1] {
  \hlOprt{#1}
}



% Param #1: the text to highlight as value

\DeclareRobustCommand{\val}[1] {
  \hlLit{#1}
}



% Param #1: the text to highlight as constant

\DeclareRobustCommand{\const}[1] {
  \hlConst{#1}
}



% Param #1: the text to highlight as function

\DeclareRobustCommand{\func}[1] {
  \hlFunc{#1}
}



% Param #1: the text to highlight as derivate function

\DeclareRobustCommand{\funcDer}[1] {
  \hlFunc{#1} \hlOprt{'}
}



% Param #1: the text to make bold, for sets

\DeclareRobustCommand{\set}[1] {
  \hlVar{\mathbb{#1}}
}



% Param #1: the text to make bold, for constant sets

\DeclareRobustCommand{\setBasic}[1] {
  \hlConst{\mathbb{#1}}
}



% Param #1: the function name
% Param #2: the variable argument to pass
% Param #3: the command for the function's level of nesting

\DeclareRobustCommand{\fnVar}[3] {
  \func{#1} #3{(} \hlVar{#2} #3{)}
}



% Param #1: the function name
% Param #2: the literal argument to pass
% Param #3: the command for the function's level of nesting

\DeclareRobustCommand{\fnVal}[3] {
  \func{#1} #3{(} \hlLit{#2} #3{)}
}



% Param #1: the function name
% Param #2: the constant argument to pass
% Param #3: the command for the function's level of nesting

\DeclareRobustCommand{\fnVal}[3] {
  \func{#1} #3{(} \hlConst{#2} #3{)}
}



% Param #1: the function name
% Param #2: the variable argument to pass

\DeclareRobustCommand{\fnVarQ}[2] {
  \func{#1} \hlLOne{(} \hlVar{#2} \hlLOne{)}
}



% Param #1: the function name
% Param #2: the literal argument to pass

\DeclareRobustCommand{\fnValQ}[2] {
  \func{#1} \hlLOne{(} \hlLit{#2} \hlLOne{)}
}



% Param #1: the function name
% Param #2: the constant argument to pass

\DeclareRobustCommand{\fnValQ}[2] {
  \func{#1} \hlLOne{(} \hlConst{#2} \hlLOne{)}
}



% Param #1: the function name
% Param #2: the code for the argument list to pass
% Param #3: the command for the function's level of nesting

\DeclareRobustCommand{\fnCustom}[3] {
  \func{#1} #3{(} #2 #3{)}
}



%% Limits



% Param #1: the approaching variable
% Param #2: the approached value

\DeclareRobustCommand{\limToVal}[2] {
  \hlOprt{\lim_{\hlVar{#1} \hlTxt{\to} \hlLit{#2}}}
}



% Param #1: the approaching variable
% Param #2: the approached variable

\DeclareRobustCommand{\limToVar}[2] {
  \hlOprt{\lim_{\hlVar{#1} \hlTxt{\to} \hlvar{#2}}}
}



% Param #1: the approaching variable
% Param #2: the approached constant

\DeclareRobustCommand{\limToConst}[2] {
  \hlOprt{\lim_{\hlVar{#1} \hlTxt{\to} \hlConst{#2}}}
}



%% Derivatives



% Param #1: the function name
% Param #2: the variable argument to pass
% Param #3: the command for the function's level of nesting

\DeclareRobustCommand{\fnDerVar}[3] {
  \funcDer{#1} #3{(} \hlVar{#2} #3{)}
}



% Param #1: the function name
% Param #2: the literal argument to pass
% Param #3: the command for the function's level of nesting

\DeclareRobustCommand{\fnDerVal}[3] {
  \funcDer{#1} #3{(} \hlLit{#2} #3{)}
}



% Param #1: the function name
% Param #2: the constant argument to pass
% Param #3: the command for the function's level of nesting

\DeclareRobustCommand{\fnDerVal}[3] {
  \funcDer{#1} #3{(} \hlConst{#2} #3{)}
}


%% Integrals



%% Complex scripts



%

\ExplSyntaxOn
\NewDocumentCommand{\argList}{m} {
  % Split the input list by commas and process each item
  \seq_set_split:Nnn \l_tmpa_seq { , } { #1 }
  \seq_map_inline:Nn \l_tmpa_seq
  {
    % Check if the item is a number
    \tl_if_blank:nF {##1} % Skip if empty (in case of extra commas)
    {
      \regex_match:nnTF { ^ \d+ (\.\d+)? $ } { ##1 }
      {
        % If numeric, apply numcolor and spacing
        {\hlLit{##1}} \, {\hlTxt{,}}\,
      }
      {
        % If not numeric, apply varcolor and spacing
        {\hlVar{##1}} \, {\hlTxt{,}}\,
      }
    }
  }
}
\ExplSyntaxOff